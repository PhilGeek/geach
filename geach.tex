%!TEX TS-program = xelatex
%!TEX TS-options = -synctex=1
%!TEX encoding = UTF-8 Unicode
%
%  geach
%
%  Created by Mark Eli Kalderon on 2011-12-26.
%  Copyright (c) 2011. All rights reserved.
%

\documentclass[12pt]{article} 

% Definitions
\newcommand\mykeywords{Frege-Geach, cognitivism, noncognitivism} 
\newcommand\myauthor{Mark Eli Kalderon} 
\newcommand\mytitle{The Philosophical Significance of the Frege-Geach Problem}

% Packages
\usepackage{geometry} \geometry{a4paper} 
\usepackage{url}
\usepackage{txfonts}
\usepackage{color}
\definecolor{gray}{rgb}{0.459,0.438,0.471}
% \usepackage{setspace}
% \doublespace % Uncomment for doublespacing if necessary
% \usepackage{epigraph} % optional

% XeTeX
\usepackage{fontspec}
\usepackage{xltxtra,xunicode}
\defaultfontfeatures{Scale=MatchLowercase,Mapping=tex-text}
\setmainfont{Hoefler Text}
\setsansfont{Gill Sans}
\setmonofont{Inconsolata}

% Section Formatting
% \usepackage[]{titlesec}
% \titleformat{\section}[hang]{\fontsize{14}{14}\scshape}{\S{\thesection}}{.5em}{}{}
% \titleformat{\subsection}[hang]{\fontsize{12}{12}\scshape}{\S{\thesubsection}}{.5em}{}{}
% \titleformat{\subsubsection}[hang]{\fontsize{12}{12}\scshape}{\S{\thesubsubsection}}{.5em}{}{}

% TODO List
% \usepackage{color}
% \usepackage{index} % use index package to create indices
% \newindex{todo}{tod}{tnd}{TODO List} % start todo list
% \newindex{fixme}{fix}{fnd}{FIXME List} % start fixme list
% \newcommand{\todo}[1]{\textcolor{blue}{TODO: #1}\index[todo]{#1}} % macro for todo entries
% \newcommand{\fixme}[1]{\textcolor{red}{FIXME: #1}\index[fixme]{#1}} % macro for fixme entries

% Bibliography
\usepackage[round]{natbib} 

% Title Information
\title{\mytitle} % For thanks comment this line and uncomment the line below
% \title{\mytitle\thanks{}}% 
\author{\myauthor} 
\date{} % Leave blank for no date, comment out for most recent date

% PDF Stuff
% \usepackage[plainpages=false, pdfpagelabels, bookmarksnumbered, backref, pdftitle={\mytitle}, pagebackref, pdfauthor={\myauthor}, pdfkeywords={\mykeywords}, xetex, colorlinks=true, citecolor=gray, linkcolor=gray, urlcolor=gray]{hyperref} 

%%% BEGIN DOCUMENT
\begin{document}

% Title Page
\maketitle
% \begin{abstract} % optional
% \noindent
% \end{abstract} 
\vskip 2em \hrule height 0.4pt \vskip 2em
% Main Content
% \epigraph{}

% Layout Settings
\setlength{\parindent}{1em}

\section{Introduction} % (fold)
\label{sec:introduction}

The Frege-Geach problem is much discussed. But despite the active interest, unclarity remains about what the problem is. Too often it is represented as a narrow technical problem facing non-representational, expressivist semantics. I have long thought that its philosophical significance runs deeper than that. The present essay is a (tentative and provisional) attempt to say why.

Fundamentally, the problem concerns the nature of judgment and is only derivatively a problem for semantics. Specifically, the Frege-Geach problem is a challenge to noncognitivism. Noncognitivism is a claim about the nature of judgment operative in a given discourse, be it moral, modal, counterfactual, or what have you. According to noncognitivism, judgment is not a cognitive attitude borne to a moral, modal, or counterfactual proposition. The operative notion of judgment may have a cognitive attitude as a constituent (even \citealt{Ayer:1946uq} did not deny that ethical claims were utterly devoid of factual content), but if it does, the constituent cognitive attitude is born to a proposition about some other subject matter. Notice that, so construed, noncognitivism is a psychological thesis \cite[as I argue in][]{Kalderon:2005gz}. Of course if our judgments about some putative subject matter are noncognitive, then statements about that subject matter had better not express belief in propositions that represent that subject matter. So the psychological thesis has semantic consequences. But the Frege-Geach problem is not fundamentally a problem about the nature of semantics; rather, it is a problem about the nature of judgment as the noncognitivist conceives of it. 

If that's right, then there is some requirement on the nature of judgment that rules out a noncognivist understanding of it. As such, I shall begin with the nature of judgment taking Geach's discussion in \emph{Mental Acts} \citeyearpar{Geach:1957ys} as my starting point. 

% section introduction (end)

\section{A Disanalogy between Thought and Speech} % (fold)
\label{sec:a_disanalogy_between_thought_and_speech}

% section a_disanalogy_between_thought_and_speech (end)

% Bibligography
\bibliographystyle{plainnat} 
\bibliography{Philosophy} 

\end{document}